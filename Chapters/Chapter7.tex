% Chapter Template

\chapter{Implementació} % Main chapter title

\label{Chapter7} % Change X to a consecutive number; for referencing this chapter elsewhere, use \ref{ChapterX}

Un cop decidit com serà el sistema, és a dir, quins requisits, objectius i abast tindrà, i quin disseny s'aplicarà en ell, el següent és parla de la implementació per si mateixa. Les decisions preses anteriorment i explicades en capítols anteriors han marcat el que s'explicarà en aquest capítol. En primera instància es realitzarà un repàs per les tecnologies i eines utilitzades durant el desenvolupament de \textit{Wisebite}, i per acabar es seguirà pels detalls de la implementació del projecte.

%----------------------------------------------------------------------------------------
%	SECTION 1
%----------------------------------------------------------------------------------------

\section{Tecnologies i eines utilitzades}

Durant el desenvolupament del projecte s'ha utilitzat tot tipus de tecnologies, és per això que s'ha decidit subdividir-ho en quatre conceptes. Començant pels llenguatges de programació, seguint per les bases de dades, mencionant les eines utilitzades i finalitzant amb les llibreries externes utilitzades.

\subsection{Llenguatge de programació}

\textbf{\large Java}\cite{java}\\
Com es va comentar anteriorment en la decisió de l'arquitectura, es va decidir construir una aplicació Android. Aquest sistema operatiu permet desenvolupar el \textit{backend} de les seves aplicacions natives en Java, C++ i Kotlin. Donat que el desenvolupador tenia més coneixement en desenvolupament d'aplicacions Android en Java, es va decidir utilitzar aquest llenguatge. A més a més, al ser el llenguatge de programació més utilitzat en el món des de ja fa anys, la comunitat d'usuaris és molt gran. En conseqüència, la probabilitat de trobar solucions a possibles problemes que es poden ocasionar és molt més alta que pas amb els altres dos llenguatges.
\\\\
\textbf{\large XML}\cite{xml}\\
El desenvolupament del \textit{frontend} en Android es realitza en llenguatge d'etiquetes, específicament XML. A diferència del \textit{backend}, en aquest cas no es tenia elecció dins del context de desenvolupament d'aplicacions natives.
\\\\
\textbf{\large \LaTeX}\cite{latex}\\
La realització de la memòria del projecte ha estat realitzada amb el llenguatge de software lliure anomenat \LaTeX. És un llenguatge que et permet realitzar documents pdf a través d'unes directives. L'esforç inicial d'aprendre aquest llenguatge es va contemplar en la planificació temporal del projecte, i tot i conèixer que seria d'un període més o menys considerable, es va contemplar l'esforç com útil en el projecte per la professionalitat i la robustesa que reflexa el document.
\\\\
\begin{figure}[H]
\centering
\includegraphics[scale=0.20]{Figures/java.png}
\includegraphics[scale=0.15]{Figures/xml.png}
\hspace{0.8cm}
\includegraphics[scale=0.20]{Figures/latex.png}
\caption{Llenguatges de programació}
\end{figure}

\subsection{Bases de dades}

TODO: Redactar

\subsection{Eines}

TODO: Redactar

\subsection{Llibreries externes}

TODO: Redactar

%----------------------------------------------------------------------------------------
%	SECTION 2
%----------------------------------------------------------------------------------------

\section{Detalls de la implementació}

TODO: Redactar

\subsection{Implementació de l'aplicació}

TODO: Redactar

\subsection{Implementació de la memòria}

TODO: Redactar
