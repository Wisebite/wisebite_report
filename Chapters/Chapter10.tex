% Chapter Template

\chapter{Gestió econòmica} % Main chapter title

\label{Chapter10} % Change X to a consecutive number; for referencing this chapter elsewhere, use \ref{ChapterX}

En aquest apartat es tractarà el cost econòmic que causarà la construcció d’aquest projecte. El treball final de grau és un projecte no remunerat, és a dir, que l’estudiant no rebrà cap recompensa econòmica per a la realització d’aquest. Tot i així, suposarem l’existència d’un equip de treball compost per un cap de projecte, un grup d’analistes i dissenyadors i un conjunt de programadors i provadors.

%----------------------------------------------------------------------------------------
%	SECTION 1
%----------------------------------------------------------------------------------------

\section{Identificació i estimació de costos}

Per a la construcció del pressupost\cite{presupuesto} d’aquest projecte es necessita tenir en compte els costos directes, que representen als recursos humans, els costos indirectes, que es reflecteixen als recursos materials, i a les continències i imprevistos. Tots ells, després de la seva anàlisi, esdevindran al pressupost final d’aquest projecte.

\subsection{Costos directes}

Els costos directes són el conjunt de despeses que corresponen al grup de recursos humans que necessita el projecte. Tot i que aquest projecte, com s’ha comentat abans, no té cap tipus de remuneració, es fa només amb afany acadèmic i és només individual, suposarem l’existència d’un equip de treball qualificat econòmicament. Per apropar-nos més al mercat s’ha assignat un responsable d’equip a cada una de les tasques reflectides al diagrama de Gantt. S’ha estimat que el cap de projecte tindrà un salari al voltant dels 45\euro/hora, el grup d’analistes i dissenyadors 35\euro/hora i l’equip de programadors i provadors de 20\euro/hora.
\\\\
Aleshores, com a resultant d’aquesta anàlisi sorgeix la taula següent amb la quantitat d’hores, cost unitari i total de cada una de les activitats del projecte.
\\\\
TODO: Adjuntar taula
\\\\
Les hores calculades per activitat s’han calculat amb el supòsit que ha d’haver-hi un treball, en mitjana, de 30 hores setmanals dedicades al projecte. Finalment, hi haurà una dedicació humana de 340 hores, amb un cost de 11.050\euro\space considerant els supòsits anteriors.

\subsection{Costos indirectes}

Durant tot el projecte apareixeran situacions quotidianes que repercutiran en el pressupost final amb connotativa indirecte com els que s’exposen a continuació.
\\\\
En primer terme, s’ha de valorar la connexió a Internet com a cost indirecte. El seu cost mensual és de 33\euro, no obstant s’ha de tenir en compte que no s’utilitzarà el 100\% del seu ús en tasques relacionades amb el treball final de grau. Per això se li atribueix un 30\% de dedicació.
\\\\
En segon lloc, s’ha de valorar el hardware utilitzat. En aquest projecte només s’utilitzarà un portàtil per al desenvolupament i construcció del projecte i, addicionalment, un telèfon mòbil amb sistema operatiu Android per a realitzar les proves pertinents segons vagi evolucionant el producte. El cost del portàtil se situa als 500\euro\space juntament amb els 310\euro\space del telèfon mòbil. No obstant això, es considera una vida útil de 4 anys per al portàtil i de 2 anys i mig per al mòbil. Per tant, se li atribueix un 10\% i un 17\% de dedicació a cada un dels dispositius tecnològics respectivament.
\\\\
Per últim no oblidar-nos de les impressions en paper que es realitzaran durant el projecte. Tant durant el desenvolupament d’aquest com en la memòria final s’imprimirà documentació. S’ha valorat que s’utilitzaran 500 impressions durant tot el projecte a 0.05\euro\space la unitat.
\\\\
TODO: Adjuntar taula


\subsection{Contingències}

En tot pressupost sempre s’ha de reservar un 15\% de continències, que s’aplicaran tant en els costos directes com indirectes.
\\\\
TODO: Adjuntar taula

\subsection{Imprevistos}

Durant el transcurs del desenvolupament del projecte és possible que apareguin imprevistos que afectin el pressupost.
\\\\
El primer es considera la mala planificiació en temps del projecte. Pot donar-se el cas en què algunes setmanes s’hagi d’invertir més temps en la programació i desenvolupament de la plataforma per complir els terminis establerts en la planificació inicial. S’ha valorat afegir unes 30 hores extres durant tot el projecte, les quals seran atribuïdes al paper de programador o provador, recordant el seu salari de 20\euro/hora. Per tant, es valora aquest cost amb 600\euro.
\\\\
L’altre possible imprevist que es pot trobar el projecte és alguna incidència amb el hardware utilitzat, és a dir, l’ordinador portàtil o el telèfon mòbil. S’ha valorat en 150\euro\space la possible reparació davant d’un presumpte problema que impedeixi el desenvolupament d’aquest projecte.
\\\\
TODO: Adjuntar taula

\subsection{Pressupost final}

Per a la realització del pressupost final no s’ha tingut en compte cap tipus de variació en els costos, com podria ser l’augment de salaris, donat que el projecte és de curta durada. Per altra banda no s’ha tingut en compte cap tipus d’amortització o marge de benefici donat que és un projecte de caràcter acadèmic sense ànim de lucre ni guanyar diners amb el resultat d’aquest.
\\\\
TODO: Adjuntar taula

%----------------------------------------------------------------------------------------
%	SECTION 2
%----------------------------------------------------------------------------------------

\section{Control de gestió}

Per a portar un bon control de si el pressupost és el correcte i tot va segons hauria d’anar es durà a terme un seguit d’accions per portar-ho tot al dia.
\\\\
El primer consisteix en el registre d’accions relacionades amb el projecte. Es mantindrà una taula amb un registre de les hores empleades en el projecte detallant en quin àmbit s’està treballant. Aquestes dades es revisaran de forma periòdica, sobretot en les retrospectives de les iteracions, per veure si el temps dedicat en cada una de les tasques és el previst anteriorment.
\\\\
Aleshores al final del projecte s’agafarà aquestes dades i es farà una anàlisi comparant-ho amb el pressupost inicial establert, calculant els costos directes i indirectes reals i els imprevistos que hi ha hagut.
\\\\
Si es veu que hi ha una desviació notable comparant la realitat amb el pressupost planejat, es prendrà cura en veure quin ha estat el punt on s’ha fallat en la planificació.