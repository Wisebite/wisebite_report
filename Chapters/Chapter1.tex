% Chapter 1

\chapter{Introducció} % Main chapter title

\label{Chapter1} % For referencing the chapter elsewhere, use \ref{Chapter1} 

%----------------------------------------------------------------------------------------

% Define some commands to keep the formatting separated from the content 
\newcommand{\keyword}[1]{\textbf{#1}}
\newcommand{\tabhead}[1]{\textbf{#1}}
\newcommand{\code}[1]{\texttt{#1}}
\newcommand{\file}[1]{\texttt{\bfseries#1}}
\newcommand{\option}[1]{\texttt{\itshape#1}}

%----------------------------------------------------------------------------------------

Aquest projecte és un Treball Final de Grau en Enginyeria Informàtica a la Facultat d'Informàtica de Barcelona (\textit{Universitat Politècnica de Catalunya}). Un projecte amb la finalitat principal de convertir un establiment de restauració qualsevol en quelcom intel·ligent, podent gestionar de forma més eficient, còmode i professional les seves comandes, podent-les analitzar posteriorment i interactuant de forma més activa amb el client de l'establiment.

\section{Contextualització}

El món de la restauració va néixer molts segles enrere i amb el transcurs de la història ha anat evolucionant proporcionalment amb l'evolució del ser humà i els seus costums. Tot i així, el que es clar és que el concepte d'anar a prendre quelcom al bar durant algun moment del dia es segueix mantenint per molt temps que passi, almenys a Espanya.
\\\\
La tecnologia ha estat quelcom que sempre ha existit, però no amb tanta importància i impacte com té actualment. Des del naixement de l'\textit{smartphone} \cite{smartphone} el 1992 quan IBM va treure el primer pilot de telèfon mòbil amb funcionalitats de PDA incorporades, s'ha anat instaurant a les nostres vides de manera exponencial fins al punt on és pràcticament una part nostra, que sense ella no seria el mateix. D'aquest aspecte se li pot treure tant punts favorables com negatius. Entre els positius tenim sistemes similars als del projecte que estem tractant, el qual dóna infinitats d'avantatges respecte el sistema convencional. Avui en dia, en el context de la societat actual en què vivim, sistemes intel·ligents implantats en els establiments de restauració es veuen a comptagotes, i no pas perquè les plataformes existents siguin dolentes o precàries, sinó per un altre seguit de factors. Factors com pot ser l'impacte econòmic que implica la instauració d'un sistema d'aquestes característiques, la manca d'adaptabilitat al canvi o bé la complexitat d'alguns establiments.
\\\\
El fet és que des de principis de segle els costums humans canvien amb una rapidesa realment diferent de la de fa segles, tan ràpidament que el món de la restauració en conjunt no s'ha pogut adaptar.
El naixement de les noves tecnologies i el poder que tenen avui en dia a la societat es veu reflectit en bars i restaurants, on algun d'ells (i cada cop més) la utilitzen en el negoci.
Tot i així, en l'estudi d'aquest tòpic, apareix un seguit de qüestions gens menyspreables, les quals han de ser tractades.

\subsection{Factor econòmic}

Un dels principals problemes per als quals aquest sector no s'ha acabat d'adaptar és el cost de la implantació de la tecnologia en un establiment d'aquestes característiques. Cada restaurant o bar és únic en referència a la resta, per tant, cada un d'ells necessita un sistema adaptat a les seves necessitats, i això es fa pagar.
\\\\
El desenvolupament d'un sistema genèric és notablement més barat respecte un específic, ja que l'equip encarregat de construir-ho pot vendre-ho posteriorment a més d'un client, així doncs pot ajustar més el preu. En canvi, si estem parlant d'un sistema totalment personalitzat i especialitzat per un establiment, llavors el cost puja considerablement, ja que han de cobrir els costos del disseny i la implementació del sistema.
\\\\
És aquí doncs on s'estableix un dels tres grans problemes que fa que sistemes d'aquestes característiques no es vegin avui en dia en els establiments de restauració. A més a més se li suma l'època de crisi econòmica viscuda que fa complicar el panorama. Només els establiments que aconsegueixen facturar grans quantitats es poden permetre sistemes com el que comentem.

\subsection{Adaptació al canvi}

En aquest sector ens podem trobar molts tipus d'usuaris. Perfils de gent que sempre busquen ser millors en el sector i fan el possible per estar actualitzats amb la tecnologia d'aquell moment. 
En canvi, existeixen nombrosos casos d'establiments on els responsables d'aquests no tenen facilitat per adaptar-se al canvi, és a dir, que se satisfan amb el procediment de negoci que sempre han tingut i sempre els ha funcionat, encara que sigui antiquat. Amb dificultats com aquestes no és fàcil instaurar un sistema d'aquestes característiques, ja que els usuaris que la utilitzen no s'adaptarien i, en conseqüència, tindria represàlies negatives.
\\\\
La implantació de tot sistema en una corporació o empresa, ja sigui enfocat en el món de la restauració o bé en un altre sector, no només recau en la compra de material \textit{hardware} i \textit{software}, sinó que també recau en la implicació del treballadors que hauran d'interactuar amb aquest nou sistema. Per tant, és obligació dels responsables de tot establiment que vulgui implantar un sistema d'aquestes característiques motivar a l'equip de treballadors en aquest aspecte. La plataforma a implantar ja pot ser molt bona, però si no hi ha voluntat de l'equip en utilitzar-la correctament, molt probablement el procés acabarà fallant. 

\subsection{Complexitat}

Tots els establiments d'aquest sector funcionen de maneres molt diferents acord a les seves característiques i funcionalitats. Alguns d'ells disposen de sistemes molt complexos i complicats en comparació de la competència, cosa que aporta dificultat en la implantació d'un sistema d'aquest tipus. I en contrapartida, implantar una plataforma d'aquestes característiques en un establiment molt simple com pot arribar a ser un bar de poble tampoc acaba de ser del tot útil.
\\\\
En conseqüència, podem tenir dos situacions que impedeixen el \textit{boom} d'aquests sistemes en el món de restauració. Per una banda, restaurants molt complexos que incapaciten crear un sistema que ho controli tot de forma fàcil. I per altra banda, bars molt simples o senzills que mai s'arribaran a plantejar sistemes d'aquestes característiques.

%----------------------------------------------------------------------------------------

\section{Motivació}

Durant el quadrimestre anterior a l'inici del Treball Final de Grau vaig estar rumiant profundament cap a on volia encaminar el projecte. Tenia disponible la capacitat de realitzar-ho en la empresa en la qual estava treballant, i actualment segueixo. Tot i així, donades unes circumstàncies que es comentarà a continuació, vaig decidir encaminar-me a realitzar el projecte de \textit{Wisebite}.
\\\\
En el meu context familiar i d'amistats he tingut sempre molt present la cultura de la gastronomia, com bé caracteritza el nostre país. Tot i així, amb els anys he anat coneixent persones que es dediquen professionalment al món de la restauració, sigui cambrers, cuiners o administradors d'establiments del sector. Al anar amb aquest tipus de persones a prendre quelcom o a menjar un àpat, em feien veure diferent l'establiment de com ho feia abans. Molts cops ens centràvem més en com era el servei i com estava muntat internament la cadena de producció dins l'establiment que pas gaudir de l'àpat que ens estaven servint.
\\\\
En conseqüència, a arrel d'això i dels coneixements tècnics que he anat adquirint durant el grau aquests quatre anys, vaig estar pensant com es podria aplicar la tecnologia en establiments d'aquest tipus per tal de millorar la seva eficiència i poder oferir un millor producte als clients.
\\\\
Per dissenyar i construir una idea més autèntica vaig estar parlant amb aquest grup de persones, que s'ha comentat anteriorment, i se'ls va preguntar com funcionaven els seus establiments i que realitzarien per poder millorar-los al que correspon a la gestió del bar o restaurant. Una gran majoria d'aquests em va comentar que van implantar un sistema de gestió de comandes a travès d'un mòbil o tauleta, però que els havia costat molt temps acabar-ho implantant donat al cost que comportava. I no només això, sinó que els hi va costar bastant adaptar-se degut a la poca usabilitat que tenia el sistema que utilitzaven.
\\\\
Així doncs, vist quin era l'estat actual, vaig decidir-me a realitzar un estudi de mercat analitzant quines aplicacions i plataformes existien en aquell moment (apartat que comentarem posteriorment) i em vaig adonar que hi havia molta feina per fer. Les aplicacions que aplicaven la filosofia de \textit{Wisebite} estaven bastant obsoletes i no disposaven d'una interfície d'usuari que tendia a la usabilitat, fet que dificultava l'adaptabilitat al canvi dels usuaris.
\\\\
En conclusió, després d'estudiar bé la proposta vaig parlar l'\textit{Ernest Teniente} i va acceptar ser el meu director d'aquest projecte i Treball Final de Grau.