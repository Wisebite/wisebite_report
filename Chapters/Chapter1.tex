% Chapter 1

\chapter{Introducció} % Main chapter title

\label{Chapter1} % For referencing the chapter elsewhere, use \ref{Chapter1} 

%----------------------------------------------------------------------------------------

% Define some commands to keep the formatting separated from the content 
\newcommand{\keyword}[1]{\textbf{#1}}
\newcommand{\tabhead}[1]{\textbf{#1}}
\newcommand{\code}[1]{\texttt{#1}}
\newcommand{\file}[1]{\texttt{\bfseries#1}}
\newcommand{\option}[1]{\texttt{\itshape#1}}

%----------------------------------------------------------------------------------------

Aquest projecte és un Treball Final de Grau en Enginyeria Informàtica a la Facultat d'Informàtica de Barcelona (\textit{Universitat Politècnica de Catalunya}). Un projecte amb la finalitat principal de convertir un establiment de restauració qualsevol en quelcom intel·ligent, podent gestionar de forma més eficient, còmode i professional les seves comandes, podent-les analitzar posteriorment i interactuant de forma més activa amb el client de l'establiment.

\section{Contextualització}

El món de la restauració va néixer molts segles enrere i amb el transcurs de la història ha anat evolucionant proporcionalment amb l'evolució del ser humà i els seus costums. Tot i així, el que es clar és que el concepte d'anar a prendre quelcom al bar durant algun moment del dia es segueix mantenint per molt temps que passi, almenys a Espanya.

La tecnologia ha estat quelcom que sempre ha existit, però no amb tanta importància i impacte com té actualment. Des del naixement de l'\textit{smartphone} \cite{smartphone} el 1992 quan IBM va treure el primer pilot de telèfon mòbil amb funcionalitats de PDA incorporades, s'ha anat instaurant a les nostres vides de manera exponencial fins al punt on és pràcticament una part nostra, que sense ella no seria el mateix.

D'aquest aspecte se li pot treure tant punts favorables com negatius. Entre els positius tenim sistemes similars als del projecte que estem tractant, el qual dóna infinitats d'avantatges respecte el sistema convencional.

Avui en dia, en el context de la societat actual en què vivim, sistemes intel·ligents implantats en els establiments de restauració es veuen a comptagotes, i no pas perquè les plataformes existents siguin dolentes o precàries, sinó per un altre seguit de factors. Factors com pot ser l'impacte econòmic que implica la instauració d'un sistema d'aquestes característiques, la manca d’adaptabilitat al canvi o bé la complexitat d'alguns establiments.

El fet és que des de principis de segle els costums humans canvien amb una rapidesa realment diferent de la de fa segles, tan ràpidament que el món de la restauració en conjunt no s'ha pogut adaptar.
El naixement de les noves tecnologies i el poder que tenen avui en dia a la societat es veu reflectit en bars i restaurants, on algun d'ells (i cada cop més) la utilitzen en el negoci.
Tot i així, en l'estudi d'aquest tòpic, apareix un seguit de qüestions gens menyspreables, les quals han de ser tractades.

\subsection{Factor econòmic}

Un dels principals problemes per als quals aquest sector no s’ha acabat d’adaptar és el cost de la implantació de la tecnologia en un establiment d’aquestes característiques. Cada restaurant o bar és únic en referència a la resta, per tant, cada un d’ells necessita un sistema adaptat a les seves necessitats, i això es fa pagar.

\subsection{Adaptació al canvi}

En aquest sector ens podem trobar molts tipus d’usuaris. Perfils de gent que sempre busquen ser millors en el sector i fan el possible per estar actualitzats amb la tecnologia d’aquell moment. 
En canvi, existeixen nombrosos casos d’establiments on els responsables d’aquests no tenen facilitat per adaptar-se al canvi, és a dir, que se satisfan amb el procediment de negoci que sempre han tingut i sempre els ha funcionat, encara que sigui antiquat. Amb dificultats com aquestes no és fàcil instaurar un sistema d’aquestes característiques, ja que els usuaris que la utilitzen no s’adaptarien i, en conseqüència, tindria represàlies negatives.

\subsection{Complexitat}

Tots els establiments d’aquest sector funcionen de maneres molt diferents acord a les seves característiques. Alguns d’ells disposen de sistemes molt més complexos i complicats que la competència, cosa que aporta dificultat en la implantació d’un sistema d’aquest tipus.

%----------------------------------------------------------------------------------------

\section{Motivació}

\LaTeX{} is not a \textsc{wysiwyg} (What You See is What You Get) program, unlike word processors such as Microsoft Word or Apple's Pages. Instead, a document written for \LaTeX{} is actually a simple, plain text file that contains \emph{no formatting}. You tell \LaTeX{} how you want the formatting in the finished document by writing in simple commands amongst the text, for example, if I want to use \emph{italic text for emphasis}, I write the \verb|\emph{text}| command and put the text I want in italics in between the curly braces. This means that \LaTeX{} is a \enquote{mark-up} language, very much like HTML.
