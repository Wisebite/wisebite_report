% Chapter Template

\chapter{Anàlisi de requisits} % Main chapter title

\label{Chapter4} % Change X to a consecutive number; for referencing this chapter elsewhere, use \ref{ChapterX}

%----------------------------------------------------------------------------------------
%	SECTION 1
%----------------------------------------------------------------------------------------

\section{Agents implicats}
En tot projecte en el qual ens podem trobar, inclòs un treball final de grau com en el que ens trobem, un conjunt o col·lectiu de persones afectades de forma directa o indirecte.
\\\\
En concret, en el que es refereix \textit{Wisebite}, destaca en especial un dels punts comentats en capítols anteriors. Dins les tres problemàtiques per les quals els usuaris no disposaven d'una implantació d'un sistema de gestió en el seu establiment de restauració és per l'elevat cost que té, és a dir, el problema ve degut per un factor econòmic. Aquest col·lectiu de persones que es veuen dins d'aquesta problemàtica no és pas un grup petit, sinó tot el contrari. És per això que és molt important valorar en aquest projecte quins són els agents implicats.
\\\\
Les parts interessades o stakeholders\cite{stakeholder} d'un projecte són aquelles persones o agrupacions de persones que el projecte els afecta de manera directa o indirecta. Aquests grups poden tenir objectius totalment diferents entre ells, i que cada part interessada jugui un paper clau en el desenvolupament i vida del projecte. Per això és important destacar cada una d'elles.

\subsection{Director del projecte}
En trobar-nos en un Treball Final de Grau, apareix la figura del director. En \textit{Wisebite}, el director és l'Ernest Teniente\cite{ernestteniente}, actualment professor de la Universitat Politècnica de Catalunya. Va ser el primer contacte quant a la inscripció del projecte i és el responsable de guiar a l'autor del projecte durant tot el transcurs d'aquest i supervisar tots els punts que vegi necessaris. Guiarà a l'autor del projecte i facilitarà tots els seus recursos disponibles per a poder construir un bon treball conjuntament amb l'autor d'aquest.

\subsection{Equip desenvolupador}
El grup de desenvolupadors del projecte són un dels actors més importants, per no dir el més important, ja que aporta la capacitat de convertir la idea teòrica a la pràctica. Al parlar d'un treball final de grau, l'equip de desenvolupadors es redueix a una sola persona, l'estudiant i autor d'aquest.
\\\\
Aquesta persona té com a objectiu iniciar i llançar el projecte endavant, passant per tot el procés d'inscripció de treball. Un cop passada aquesta etapa, l'equip desenvolupador ha de dissenyar i perfeccionar la idea, definir-la, implementar-la i documentar-la per així poder-la presentar en la fase final del treball.

\subsection{Establiments de restauració}
Com s'ha comentat anteriorment, l'aparició de \textit{Wisebite} té com a objectiu principal convertir i fer evolucionar el món de la restauració. L'aparició de sistemes com aquest aporta un gran canvi a aquest sector. Els responsables de cada un dels establiments disponibles al mercat tindran la possibilitat d'implantar aquest projecte al seu negoci amb l'objectiu de millorar els resultats.
\\\\
El paper, i la reacció que esdevingui d'aquest col·lectiu, en conèixer l'aparició de \textit{Wisebite} serà de gran importància per definir el futur de la plataforma. Ens podem trobar amb la situació que l'aplicació guanyi gran fama i es faci un bon lloc dins d'aquest sector, o bé oposadament que acabi passant desapercebuda dins de la restauració. Aquest futur es decideix en la reacció d'aquest col·lectiu.

\subsection{Clients}
Si un establiment, sigui bar o restaurant, decideix implantar aquest sistema, no només canviarà la perspectiva de l'empleat sinó també del client o usuari. El comportament que realitzava anteriorment en entrar a aquest establiment haurà de canviar una mica per adaptar-se al nou sistema implantat.
\\\\
En primera instància, podrà observar com la gestió de comandes dels cambrers ha canviat de forma dràstica i que tota la gestió del restaurant en general ha evolucionat. Per altra banda, com s'ha mencionat en l'abast del projecte, tu com a usuari de \textit{Wisebite} no et fa falta pertànyer a un restaurant determinat en utilitzar l'aplicació, sinó que també pots interactuar amb la plataforma amb el perfil de client, és a dir, cercant els restaurants de la zona, consultar els seus detalls i fins i tot poder realitzar comandes des del seu propi terminal a una taula especificada.
\\\\
En conseqüència, aquí els clients pertanyents a aquests establiments de restauració són un altre col·lectiu molt important en el transcurs de la història de \textit{Wisebite}.

\subsection{Competència}
Els propietaris i responsables de sistemes similars al d'aquest projecte veuran amenaçada la seva idea i projecció de negoci, propietaris com els de les plataformes i aplicacions comentades en l'estudi de mercat de capítols anteriors.
\\\\
Aquests altres sistemes poden comportar-se de dues maneres diferents i oposades entre si. Per una banda, podrien aportar millores als seus respectius projectes per així oferir una millor plataforma als usuaris d'aquesta, ja sigui des del perfil treballador com client. O bé per altra banda ignorar-ho i mantenir el seu pla de negoci. Fet que podria fer destacar \textit{Wisebite} com la diferent dins del sector i fer-la important en aquest.
\\\\
Així doncs, igual que els altres col·lectius, aquest té una importància diferent però igual d'important pel paper que hi juga en l'evolució i futur de la plataforma.


%----------------------------------------------------------------------------------------
%	SECTION 2
%----------------------------------------------------------------------------------------

\section{Requisits funcionals}

TODO: Redactar

%----------------------------------------------------------------------------------------
%	SECTION 3
%----------------------------------------------------------------------------------------

\section{Requisits no funcionals}

En aquest apartat es realitzarà un repàs de tots els requisits no funcionals\cite{requisito} o de qualitat dels quals disposa el sistema, és a dir, l'especificació de quelcom sobre el mateix sistema, i de com s'ha de realitzar les accions pertinents a aquest. Per entendre-ho amb més facilitat s'ha dividit el conjunt de requisits segons el tipus descrit per \textit{Volere}\cite{volere}.
\\\\
% ONE
\noindent\textbf{Requisits d'aparença}
\begin{itemize}
\item \textit{Tipus de requisit (Volere)}: 10a
\item \textit{Descripció}: Disseny atractiu i d'ús senzill que convidarà a l'usuari a fer-ne ús amb més facilitat.
\item \textit{Justificació del requisit}: Com l'aplicació treballa sobre un nombre de persones considerable, com és tot el sector de la restauració i els seus respectius clients, cal destacar per l'atractiu de l'aplicació per tal de marcar un abans i un després en l'usuari, és a dir, que gaudeixi de l'experiència a l'hora d'utilitzar \textit{Wisebite}.
\item \textit{Condició de satisfacció}: El requisit se satisfarà si s'obté una bona valoració dels usuaris en respecte a l'aparença. Es podrà verificar amb una enquesta de satisfacció al conjunt d'usuaris de l'aplicació.
\end{itemize}

% TWO
\noindent\textbf{Requisits d'estil}
\begin{itemize}
\item \textit{Tipus de requisit (Volere)}: 10b
\item \textit{Descripció}: Disseny modern i ambiciós, seguint la tendència en disseny però destacant en punts específics.
\item \textit{Justificació del requisit}: La competència del mercat ens obliga a disposar d'un disseny modern per tal destacar sobre la comunitat d'usuaris, i així guanyar usuaris no només per les funcionalitats de la plataforma, sinó per l'estil de l'aplicació.
\item \textit{Condició de satisfacció}: El requisit se satisfarà si més de tres quartes parts dels usuaris consideren que \textit{Wisebite} disposa d'un disseny modern, fet que es podrà comprovar amb una enquesta.
\end{itemize}

% THREE
\noindent\textbf{Requisits de facilitat d'ús}
\begin{itemize}
\item \textit{Tipus de requisit (Volere)}: 11a
\item \textit{Descripció}: El sistema ha de ser intuïtiu i fàcil d'usar. Complirà els criteris en temes de disseny, de contingut, d'estructura i de presentació fixats pel W3C\cite{w3c}.
\item \textit{Justificació del requisit}: Un punt diferenciador important és que l'usuari pugui fer servir el sistema intuïtivament, de manera que no perdi el temps intentant descobrir com funciona i, a més a més, que qualsevol persona sigui capaç de familiaritzar-se amb el sistema. Aquest fet és molt important, ja que la majoria dels usuaris de \textit{Wisebite} no formarà part de la comunitat dels informàtics.
\item \textit{Condició de satisfacció}: El requisit se satisfarà si un usuari amb poca experiència en aplicacions aconsegueix usar-lo sense cap problema. Per això, s'utilitzarà un grup de persones inexpert en l'ús d'aplicacions mòbil per veure quina és la seva reacció utilitzant \textit{Wisebite}.
\end{itemize}

% FOUR
\noindent\textbf{Requisits de latència i velocitat}
\begin{itemize}
\item \textit{Tipus de requisit (Volere)}: 12a
\item \textit{Descripció}: La resposta del sistema ha de ser de menys d'un segon com a mínim en el 95\% de les operacions.
\item \textit{Justificació del requisit}: Un temps de resposta ràpid permet que l'usuari no perdi el flux o atenció del que està fent amb el sistema. Una plataforma de latència i velocitat dolenta produiria insatisfacció per part de l'usuari.
\item \textit{Condició de satisfacció}: El requisit se satisfarà si donat un estudi sobre el rendiment de l'aplicació, aquest confirma que el temps d'espera en cada acció és menor al segon en el 95\% dels casos.
\end{itemize}

% FIVE
\noindent\textbf{Requisits de precisió o exactitud}
\begin{itemize}
\item \textit{Tipus de requisit (Volere)}: 12c
\item \textit{Descripció}: Totes les dates que s'incloguin en l'aplicació tindran el format universal: \textit{DD/MM/AAAA}
\item \textit{Justificació del requisit}: És convenient especificar el format de la data, ja que no a tot arreu té el mateix format i podria provocar malentesos i confusions entre els usuaris.
\item \textit{Condició de satisfacció}: El requisit se satisfarà si el format de la data i l'hora segueix l'estàndard ISO-8601\cite{iso8601} extens d'estil Europeu (EN 28601).
\end{itemize}

% SIX
\noindent\textbf{Requisit de disponibilitat}
\begin{itemize}
\item \textit{Tipus de requisit (Volere)}: 12d
\item \textit{Descripció}: El sistema haurà d'estar disponible les 24 hores del dia durant els 365 dies que conformen l'any.
\item \textit{Justificació del requisit}: Els usuaris han de poder utilitzar el sistema en qualsevol moment del dia per tal de poder buscar restaurants o gestionar-los.
\item \textit{Condició de satisfacció}: El requisit se satisfarà si el sistema està disponible i completament funcional tot el temps.
\end{itemize}

% SEVEN
\noindent\textbf{Requisits d'adaptabilitat}
\begin{itemize}
\item \textit{Tipus de requisit (Volere)}: 14c
\item \textit{Descripció}: L'aplicació mòbil ha de poder-se veure i executar correctament en els diferents smartphones del mercat, i tenir les mateixes funcionalitats i característiques en tots ells.
\item \textit{Justificació del requisit}: L'existència de tants telèfons mòbils diferents i tantes versions d'Android disponibles actualment obliga a garantir que com a mínim es veurà de forma correcta i es podrà executar totes les funcionalitats de la plataforma.
\item \textit{Condició de satisfacció}: El requisit se satisfarà si el sistema es pot visualitzar i executar correctament en els principals smartphones del mercat.
\end{itemize}

% EIGHT
\noindent\textbf{Requisit d'immunitat}
\begin{itemize}
\item \textit{Tipus de requisit (Volere)}: 15e
\item \textit{Descripció}: El sistema està protegit d'atacs externs i infeccions per software maliciós.
\item \textit{Justificació del requisit}: S'ha de garantir la seguretat per evitar posar en risc la disponibilitat del sistema i la privadesa de les dades dels usuaris. Avui en dia que tot està tan digitalitzat, és un fet molt important per garantir la comoditat de l'usuari a l'hora d'accedir a la plataforma.
\item \textit{Condició de satisfacció}: El requisit se satisfarà si s'implementa la normativa de seguretat internacional ISO-17799\cite{iso17799} per tal de garantir la seguretat davant d'atacs externs.
\end{itemize}

% NINE
\noindent\textbf{Requisits legals}
\begin{itemize}
\item \textit{Tipus de requisit (Volere)}: 17a
\item \textit{Descripció}: S'aconseguiran tots els drets sobre els serveis externs que s'utilitzin a l'aplicació i a la vegada es compliran les lleis sobre el tractament de dades personals.
\item \textit{Justificació del requisit}: Es pactaran acords amb totes les empreses de les quals s'utilitzen els seus serveis, arribant a acords sigui amb la Universitat per poder aprofitar la seva plataforma o amb empreses externes. I també mostrar transparència a l'hora de no compartir dades personals per fins no vinculants al sistema.
\item \textit{Condició de satisfacció}: El requisit se satisfarà si no es rep cap denuncia per part de cap servei extern, ni de cap usuari per ús indegut de les dades personals.
\end{itemize}


%----------------------------------------------------------------------------------------
%	SECTION 4
%----------------------------------------------------------------------------------------

\section{Casos d'ús}

TODO: Redactar