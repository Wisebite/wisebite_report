% Chapter Template

\chapter{Anàlisi de requisits} % Main chapter title

\label{Chapter4} % Change X to a consecutive number; for referencing this chapter elsewhere, use \ref{ChapterX}

%----------------------------------------------------------------------------------------
%	SECTION 1
%----------------------------------------------------------------------------------------

\section{Agents implicats}

Aleshores, partint de la premissa anterior referent a l’impacte econòmic, podem extreure la conclusió que un gran número de possibles usuaris d’aquest tipus de sistemes no l’acaben implantant a causa de l’elevat cost que comporta. Per tant, aquest projecte no va adreçat a un petit grup de clients que han demanat unes necessitats concretes i tancades, sinó a un mercat. En conseqüència, el sistema té com a objectiu principal satisfer a la major part dels usuaris d’aquest sector.
\\\\
Les parts interessades o stakeholders\cite{stakeholder} d’un projecte són aquelles persones o agrupacions de persones que el projecte els afecta de manera directe o indirecta. Aquests grups poden tenir objectius totalment diferents entre ells, i cada part interessada jugar un paper clau en el desenvolupament i vida del projecte. Per això és important destacar cada una d’elles.

\subsection{Director del projecte}
El director del projecte és l’Ernest Teniente\cite{ernestteniente}, actualment professor de la Universitat Politècnica de Catalunya. Ell és el responsable de guiar a l’autor del projecte i supervisar tots els punts que vegi necessaris.

\subsection{Equip desenvolupador}
El grup de desenvolupadors del projecte són un dels actors més importants, ja que aporta la capacitat de convertir la idea teòrica a la pràctica. Al parlar d’un treball final de grau, l’equip de desenvolupadors es redueix a una sola persona, l’estudiant i autor d’aquest.

\subsection{Establiments de restauració}
L’aparició de sistemes com aquest aporta un gran canvi a aquest sector. Els responsables de cada un dels establiments disponibles al mercat tindran la possibilitat d’implantar aquest projecte al seu negoci amb l’objectiu de millorar els resultats.

\subsection{Clients}
Si un establiment, sigui bar o restaurant, decideix implantar aquest sistema, no només canviarà la perspectiva de l’empleat sinó també del client o usuari. El comportament que realitzava anteriorment en entrar a aquest establiment haurà de canviar una mica per adaptar-se al nou sistema implantat.

\subsection{Competència}
Els propietaris i responsables de sistemes similars al d’aquest projecte veuran amenaçada la seva idea i projecció de negoci. Aquests altres sistemes podrien aportar millores als seus respectius projectes per així oferir un millor producte als usuaris d’aquest, ja sigui des del perfil treballador com client.


%----------------------------------------------------------------------------------------
%	SECTION 2
%----------------------------------------------------------------------------------------

\section{Requisits funcionals}

TODO: Redactar

%----------------------------------------------------------------------------------------
%	SECTION 3
%----------------------------------------------------------------------------------------

\section{Requisits no funcionals}

TODO: Redactar

%----------------------------------------------------------------------------------------
%	SECTION 4
%----------------------------------------------------------------------------------------

\section{Casos d'ús}

TODO: Redactar