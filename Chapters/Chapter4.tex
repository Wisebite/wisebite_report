% Chapter Template

\chapter{Anàlisi de requisits} % Main chapter title

\label{Chapter4} % Change X to a consecutive number; for referencing this chapter elsewhere, use \ref{ChapterX}

%----------------------------------------------------------------------------------------
%	SECTION 1
%----------------------------------------------------------------------------------------

\section{Agents implicats}
En tot projecte en el qual ens podem trobar, inclòs un treball final de grau com en el que ens trobem, un conjunt o col·lectiu de persones afectades de forma directa o indirecte.
\\\\
En concret, en el que es refereix \textit{Wisebite}, destaca en especial un dels punts comentats en capítols anteriors. Dins les tres problemàtiques per les quals els usuaris no disposaven d'una implantació d'un sistema de gestió en el seu establiment de restauració és per l'elevat cost que té, és a dir, el problema ve degut per un factor econòmic. Aquest col·lectiu de persones que es veuen dins d'aquesta problemàtica no és pas un grup petit, sinó tot el contrari. És per això que és molt important valorar en aquest projecte quins són els agents implicats.
\\\\
Les parts interessades o stakeholders\cite{stakeholder} d'un projecte són aquelles persones o agrupacions de persones que el projecte els afecta de manera directa o indirecta. Aquests grups poden tenir objectius totalment diferents entre ells, i que cada part interessada jugui un paper clau en el desenvolupament i vida del projecte. Per això és important destacar cada una d'elles.

\subsection{Director del projecte}
En trobar-nos en un Treball Final de Grau, apareix la figura del director. En \textit{Wisebite}, el director és l'Ernest Teniente\cite{ernestteniente}, actualment professor de la Universitat Politècnica de Catalunya. Va ser el primer contacte quant a la inscripció del projecte i és el responsable de guiar a l'autor del projecte durant tot el transcurs d'aquest i supervisar tots els punts que vegi necessaris. Guiarà a l'autor del projecte i facilitarà tots els seus recursos disponibles per a poder construir un bon treball conjuntament amb l'autor d'aquest.

\subsection{Equip desenvolupador}
El grup de desenvolupadors del projecte són un dels actors més importants, per no dir el més important, ja que aporta la capacitat de convertir la idea teòrica a la pràctica. Al parlar d'un treball final de grau, l'equip de desenvolupadors es redueix a una sola persona, l'estudiant i autor d'aquest.
\\\\
Aquesta persona té com a objectiu iniciar i llançar el projecte endavant, passant per tot el procés d'inscripció de treball. Un cop passada aquesta etapa, l'equip desenvolupador ha de dissenyar i perfeccionar la idea, definir-la, implementar-la i documentar-la per així poder-la presentar en la fase final del treball.

\subsection{Establiments de restauració}
Com s'ha comentat anteriorment, l'aparició de \textit{Wisebite} té com a objectiu principal convertir i fer evolucionar el món de la restauració. L'aparició de sistemes com aquest aporta un gran canvi a aquest sector. Els responsables de cada un dels establiments disponibles al mercat tindran la possibilitat d'implantar aquest projecte al seu negoci amb l'objectiu de millorar els resultats.
\\\\
El paper, i la reacció que esdevingui d'aquest col·lectiu, en conèixer l'aparició de \textit{Wisebite} serà de gran importància per definir el futur de la plataforma. Ens podem trobar amb la situació que l'aplicació guanyi gran fama i es faci un bon lloc dins d'aquest sector, o bé oposadament que acabi passant desapercebuda dins de la restauració. Aquest futur es decideix en la reacció d'aquest col·lectiu.

\subsection{Clients}
Si un establiment, sigui bar o restaurant, decideix implantar aquest sistema, no només canviarà la perspectiva de l'empleat sinó també del client o usuari. El comportament que realitzava anteriorment en entrar a aquest establiment haurà de canviar una mica per adaptar-se al nou sistema implantat.
\\\\
En primera instància, podrà observar com la gestió de comandes dels cambrers ha canviat de forma dràstica i que tota la gestió del restaurant en general ha evolucionat. Per altra banda, com s'ha mencionat en l'abast del projecte, tu com a usuari de \textit{Wisebite} no et fa falta pertànyer a un restaurant determinat en utilitzar l'aplicació, sinó que també pots interactuar amb la plataforma amb el perfil de client, és a dir, cercant els restaurants de la zona, consultar els seus detalls i fins i tot poder realitzar comandes des del seu propi terminal a una taula especificada.
\\\\
En conseqüència, aquí els clients pertanyents a aquests establiments de restauració són un altre col·lectiu molt important en el transcurs de la història de \textit{Wisebite}.

\subsection{Competència}
Els propietaris i responsables de sistemes similars al d'aquest projecte veuran amenaçada la seva idea i projecció de negoci, propietaris com els de les plataformes i aplicacions comentades en l'estudi de mercat de capítols anteriors.
\\\\
Aquests altres sistemes poden comportar-se de dues maneres diferents i oposades entre si. Per una banda, podrien aportar millores als seus respectius projectes per així oferir una millor plataforma als usuaris d'aquesta, ja sigui des del perfil treballador com client. O bé per altra banda ignorar-ho i mantenir el seu pla de negoci. Fet que podria fer destacar \textit{Wisebite} com la diferent dins del sector i fer-la important en aquest.
\\\\
Així doncs, igual que els altres col·lectius, aquest té una importància diferent però igual d'important pel paper que hi juga en l'evolució i futur de la plataforma.


%----------------------------------------------------------------------------------------
%	SECTION 2
%----------------------------------------------------------------------------------------

\section{Requisits funcionals}

TODO: Redactar

%----------------------------------------------------------------------------------------
%	SECTION 3
%----------------------------------------------------------------------------------------

\section{Requisits no funcionals}

TODO: Redactar

%----------------------------------------------------------------------------------------
%	SECTION 4
%----------------------------------------------------------------------------------------

\section{Casos d'ús}

TODO: Redactar