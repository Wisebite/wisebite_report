% Chapter Template

\chapter{Conclusions} % Main chapter title

\label{Chapter12} % Change X to a consecutive number; for referencing this chapter elsewhere, use \ref{ChapterX}

Després d'haver explicat tot respecte al projecte de \textit{Wisebite}, començant per la formulació problema, seguint pel disseny d'aquest, prosseguint amb la implementació i definint els costos temporals, econòmics i sostenibles de l'aplicació, arribem a la finalització del treball final de grau. En aquest últim capítol es detallarà el resultat obtingut, s'analitzarà en una retrospectiva i s'especularà en quin serà el futur de \textit{Wisebite}.

%----------------------------------------------------------------------------------------
% SECTION 1
%----------------------------------------------------------------------------------------

\section{Resultat}

En la fase inicial es va detallar un seguit de funcionalitats que havia de tenir l'aplicació i que es tenia com a objectiu per a la fase final d'aquest. Ara com ara, i ja amb l'aplicació implementada, es pot afirmar que s'ha arribat al final amb tots els aspectes de l'abast complet.
\\\\
L'aplicació \textit{Wisebite}, que està disponible per tot el món al \textit{Google Play}\cite{wisebite_googleplay}, compleix amb els objectius preestablerts i podria ser utilitzada per qualsevol establiment de restauració. Tot i així, es considera que un projecte mai té data de finalització, és a dir, que sempre se'l pot nodrir amb més funcionalitats. Malgrat aquest aspecte, ara com ara, \textit{Wisebite} disposa de les següents funcionalitats:
\\
\begin{itemize}
\item Crear un restaurant amb tota la seva informació bàsica, horaris d'apertura durant la setmana i els seus respectius plats i menús.
\item Crear comandes, a on s'especifica el nombre de taula del restaurant.
\item Mostra les comandes actives, considerant dues de les perspectives que disposa l'aplicació, és a dir, cambrer i cuiner.
\item Veure els detalls del seu restaurant, podent consultar tota la informació un desitja.
\item Afegir un altre usuari al seu restaurant.
\item Estudi de les estadístiques del teu restaurant amb l'ajuda d'algunes gràfiques i dades d'utilitat per a poder realitzar una anàlisi d'aquest i millorar la qualitat de l'establiment.
\item Llistar la resta dels restaurants en l'aplicació, per així actuar com a client.
\item Crear una comanda a un restaurant aliè i veure el seu estat en temps real, visualitzant quan està preparat, entregat i pagat.
\item Valoració dels plats i menús i del restaurant en si després de la creació de la comanda en qüestió.
\item Veure els comentaris i les puntuacions d'altres usuaris sobre tots els restaurants i els seus respectius plats i menús.
\item Personalitzar a gust de l'usuari el seu perfil editant la informació bàsica i la imatge de perfil.
\end{itemize}

\noindent Així doncs, es pot considerar que la plataforma compleix amb els tres punts que es va indicar a l'abast del projecte. L'aplicació pot gestionar internament l'establiment, així eliminant el bolígraf i el paper. La plataforma permet analitzar les dades emmagatzemades i treure-li un profit. I a més a més, com a client de qualsevol establiment de restauració que implementi \textit{Wisebite} pot interactuar amb ell realitzant comanda, seguint-la en temps real i valorar-la, així contribuint a la comunitat.

%----------------------------------------------------------------------------------------
%	SECTION 2
%----------------------------------------------------------------------------------------

\section{Retrospectiva}

Un cop acabat el projecte, m'agradaria ficar èmfasi en un conjunt de punts pel qual recordaré aquest treball amb bon gust de boca.
\\\\
En primer lloc, remarcar el fet que aquest treball ha estat un projecte personal que se'm va ocórrer durant l'estiu passat. Tot i tenir l'opció de realitzar el treball final de grau en l'empresa que estava treballant, i que encara hi pertanyo, i permetrem anar més relaxat durant el quadrimestre, faig decidir llançar-me a la piscina i atrevir-me amb el projecte de \textit{Wisebite}. Ara, si ho miro amb retrospectiva, tot l'esforç dedicat en ell ha contribuït de manera positiva en molts aspectes en la meva carrera personal i professional.
\\\\
Per altra banda, m'agradaria ressaltar la metodologia que s'ha posat en pràctica en el projecte i el disseny de software que ha estat implementat. Un dels fets que més m'agrada personalment de l'especialitat elegida són justament els dos mencionats, i poder-los utilitzar ambdós en la finalització del meu grau és quelcom d'agrair. Tant és el gust que tinc per aquests dos aspectes que potser finalment m'acabo especialitzant en ells en un hipotètic master a cursar.
\\\\
Així doncs, després d'aquests quatre mesos d'esforç en poder tenir \textit{Wisebite} complet i haver satisfet els objectius marcats a inicis de quadrimestre han valgut molt la pena. Acabo el grau universitari amb la satisfacció d'haver fet bé la feina.

%----------------------------------------------------------------------------------------
%	SECTION 3
%----------------------------------------------------------------------------------------

\section{Futur}

Durant el transcurs d'aquests quatre mesos en els quals he estat dedicant-me a la construcció de \textit{Wisebite}, companys de la universitat em preguntaven quina era la temàtica del meu treball final de grau, i un cop els hi explicava gran part d'aquest conjunt es sorprenia i em preguntava que si tenia intenció de llançar-ho com a \textit{startup} en finalitzar-lo.
\\\\
Aquest aspecte mai m'ho havia plantejat d'un bon inici, i de fet encara ho poso en dubte si seria possible. Però el fet que algun professor de la facultat em comentés que la idea era molt bona per tirar-la endavant com a empresa, em va motivar bastant en aquest aspecte.
\\\\
Així doncs, tot i encara no ser una idea molt segura, es té la intenció de millorar el projecte durant aquest estiu i optimitzar-lo per tenir-ho llest a principis de setembre. En aquell moment, en cas de disposar d'un \textit{Wisebite} llest per ser llançat al mercat, es buscarà com provar-ho en restaurants de confiança per veure la reacció dels membres de l'establiment. I ja doncs, qui sap que ens proporcionarà el futur.

%----------------------------------------------------------------------------------------
%	SECTION 4
%----------------------------------------------------------------------------------------

\section{Aprenentatge final}

Com a conclusió, m'agradaria emfatitzar en el concepte més important que he après després d'haver finalitzat aquest treball final de grau, i en conseqüència, el grau en enginyeria informàtica. Un professor dels que m'he anat trobant durant el transcurs del grau un dia em va comentar un fet que em va marcar en la meva manera de pensar. Em va fer reflexionar sobre el paper que té un enginyer informàtic a la societat.
\\\\
Aquests últims anys la tecnologia ha evolucionat de tal manera que té el poder de generar necessitats a la societat, és a dir, té la capacitat de canviar el comportament de l'ésser humà en el seu dia quotidià. I qui ho aconsegueix això? Els enginyers informàtics que llancen a la llum aquests projectes que fan canviar la manera de veure el món.
\\\\
Així doncs, amb un possible èxit de \textit{Wisebite} estaríem canviant totalment el món de la restauració. I tot això amb una simple plataforma. Ja que, tot i que no ho pensem, els informàtics tenim el poder de canviar el món.
\\
\begin{center}
\textit{FI.}
\end{center}