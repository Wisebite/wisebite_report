% Chapter Template

\chapter{Sostenibilitat i compromís social} % Main chapter title

\label{Chapter11} % Change X to a consecutive number; for referencing this chapter elsewhere, use \ref{ChapterX}

En tot projecte, inclòs el de final de grau, es disposa d’una valoració en el que respecta l’economia, la sociabilitat i el medi ambient del projecte. Cada un d’aquests caràcters tindrà associada una valoració, que és explicada posteriorment, que totes juntes esdevindran a la matriu de sostenibilitat\cite{impacto}.

%----------------------------------------------------------------------------------------
%	SECTION 1
%----------------------------------------------------------------------------------------

\section{Sostenibilitat econòmica}

En aquest projecte s’ha realitzat un pressupost on inclou una bona avaluació de costos, incloent-hi recursos materials i humans repartits proporcionalment per cada una de les tasques representades al diagrama de Gantt. En aquest pressupost se li ha afegit també el cost per ajustaments, actualitzacions i reparacions possibles durant la vida del projecte, cosa que el converteix en competitiu en un escenari de producte real.
\\\\
Tot i que el pressupost ha acabat tenint un valor no massa alt, es pot obtenir un menor cost retallant, filant prim o col·laborant amb altres entitats acadèmiques o professionals. No obstant això, ens exposem als riscos que això comporta. 
\\\\
Aleshores, després d’haver realitzat l’estudi econòmic del projecte i haver obtingut un pressupost resultant d’aquest, es pot valorar aquest projecte com notablement bo econòmicament. Finalment ha sorgit un cost, comptant cada un dels detalls, prou baix pels objectius que es vol complir.
\\\\
En conseqüència, el caràcter econòmic del projecte té una valoració de 7.

%----------------------------------------------------------------------------------------
%	SECTION 2
%----------------------------------------------------------------------------------------

\section{Sostenibilitat social}

Com s’ha comentat anteriorment estem en una situació a on, tot i disposar d’alta tecnologia, el món de la restauració no s’acaba de modernitzar. I gran culpa d’això ho té l’elevat cost d’implantar aquesta tecnologia en aquest tipus d’establiments donada la personalització que necessiten. És per això que l’aparició de \textit{Wisebite} pot solucionar aquesta problemàtica.
\\\\
Un dels objectius primordials d’aquest projecte és realitzar un canvi notable en el caràcter social de les persones que treballen o tenen alguna relació amb el món gastronòmic o de la restauració. És per això que, després d’una bona anàlisi, s’extreu com a conclusió que és un punt molt important en el desenvolupament d’aquesta plataforma, ja que el producte final pot esdevenir a molts canvis en el dia a dia de les persones relacionades amb aquest món. No obstant això, cal destacar que no existeix una necessitat alta en l’existència d’aquest tipus de sistemes. Tot i així, l’aparició de \textit{Wisebite} com capgirar la situació i oferir un servei que qualsevol establiment de restauració, sense excepció, pot utilitzar.
\\\\
En conseqüència, el caràcter social rep una valoració de 9.

%----------------------------------------------------------------------------------------
%	SECTION 3
%----------------------------------------------------------------------------------------

\section{Sostenibilitat ambiental}

Aquest projecte no abarca el caràcter ambiental de la matriu de sostenibilitat, ja que no és el seu objectiu. Per tant converteix l’estimació d’aquest punt en una tasca complicada.
\\\\
No obstant això, cal destacar que el poc ús de material contaminant fa que la petjada ecològica no augmenti. Tot i que el projecte no té cap incentiu ni motivació en voler-la reduir. Per tant, ni s’inverteix temps ni recursos en voler-la disminuir o augmentar.
\\\\
Per tant, se li atribueix una valoració de 7.

%----------------------------------------------------------------------------------------
%	SECTION 4
%----------------------------------------------------------------------------------------

\section{Taula de sostenibilitat}

\begin{center}
    \begin{tabular}{ | l | l |}
    \hline
	\textbf{CARÀCTER}								&\textbf{VALORACIÓ}		\\ \hline
    Econòmic										& 7						\\ \hline
    Social											& 9						\\ \hline
    Ambiental										& 7						\\ \hline
    \textbf{TOTAL}					 				& 23					\\ 
    \hline
    \end{tabular}
\end{center}