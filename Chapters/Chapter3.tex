% Chapter Template

\chapter{Definició de l'abast} % Main chapter title

\label{Chapter3} % Change X to a consecutive number; for referencing this chapter elsewhere, use \ref{ChapterX}

Un sistema d'aquestes característiques podria tenir infinitat de requisits i funcionalitats, és per això que és important definir l'abast d'aquest projecte i veure quins són els objectius d'aquest.

%----------------------------------------------------------------------------------------
%	SECTION 2
%----------------------------------------------------------------------------------------

\section{Objectius}

L'objectiu principal d'aquest Treball Final de Grau és dissenyar i construir un sistema que aconsegueixi fer la vida més fàcil als empleats de qualsevol establiment de restauració, sigui bar o restaurant, i crear una millor experiència per a qualsevol usuari d'aquests locals, és a dir, millorar la seva estança oferint-li eficiència en el servei i millor tracte per part de l'establiment.
\\\\
Com ja s'ha comentat anteriorment, l'elevat cost d'implantar un sistema com aquest és la dificultat més gran. És per això que un altre objectiu important és poder construir una plataforma plenament genèrica què la puguis personalitzar al teu gust i, en definitiva, fer-la teva. Així aconseguiràs reduir una quantitat abismal en costos i obtenir majors beneficis gràcies a les funcionalitats d'aquest sistema. Perquè com bé s'ha comentat abans el cost d'implantar un sistema d'aquestes característiques no resideix en la compra d'elements tecnològics com dispositius mòbils, sinó en la compra d'un software especialitzat.
\\\\
Per altra banda es buscarà oferir la millor experiència d'usuari possible de tal manera que el cost d'aprendre a utilitzar aquesta aplicació sigui el mínim.
\\\\
Un cop especificada, dissenyada i implementada la plataforma de \textit{Wisebite} s'intentarà llançar a establiments del cercle familiar i d'amistats per així rebre un feedback de quin és el funcionament de l'aplicació en una situació real. I finalment, si la plataforma té un bon impacte, estudiar la viabilitat econòmica del projecte.

%----------------------------------------------------------------------------------------
%	SECTION 3
%----------------------------------------------------------------------------------------

\section{Abast}

En el capítol anterior s'ha analitzat les funcionalitats de les aplicacions existents en el mercat actual. En realitzar-ho, s'ha vist que totes elles es podien classificar en tres grans grups. El sistema final estarà format per aquests tres components molt importants que li donaran valor al producte resultant, i marcarà la diferència respecta la competència del mercat, tal com s'ha comentat en l'estudi anterior.

\subsection{Gestió de comandes}

En primer terme, el sistema serà capaç de gestionar les comandes d'un establiment de restauració. La plataforma tindrà la capacitat de crear menús i plats personalitzats, amb les preferències i opcions desitjades. Cadascun d'aquest contindrà informació vital i d'importància com el preu i una petita descripció del seu contingut. Com a empleat de l'establiment que té implantat \textit{Wisebite}, podrà crear comandes fàcilment i sense cap problemàtica, amb l'ajut d'una interfície molt intuïtiva. Totes les peticions seran rebudes de forma automàtica a cuina amb una interfície còmoda i agradable per tal d'agilitzar el procés el màxim possible. Aquesta informació s'anirà actualitzant en temps real sense necessitat d'actualitzar-ho manualment.
\\\\
Un cop implantat aquest component del sistema s'estarà aconseguint una millora notable en l'eficiència de les comandes. Això provocarà una satisfacció per part de la clientela, ja que rebran les comandes sol·licitades abans, que esdevindrà a uns majors ingressos per a l'establiment donat que acudirà més gent gràcies a la possible fama que es pugui generar per aquest fet.

\subsection{Anàlisi de l'establiment}

L'avantatge més important, i amb diferència, d'emmagatzemar les dades digitalment és la facilitat de realitzar un estudi detallat d'aquestes dades. El sistema tindrà la capacitat de convertir aquestes dades sense massa significat a priori a una font d'informació que serà de gran utilitat per als responsables de l'establiment.
\\\\
Per què són tan importants les dades?, es podria arribar a preguntar qualsevol. Anteriorment es prenien decisions molt importants a partir de l'experiència de les persones i de la percepció de negoci. En canvi, amb les dades en el nostre poder, es poden prendre decisions objectives i no subjectives, ja que es prenen via dades reals del consumidor. És a dir, coneixem més al nostre client.
\\\\
De forma periòdica, el sistema reportarà resums on es reflectirà informació de gran valor per a l'establiment. Informació com pot ser el tràfic setmanal, els plats més demanats, ingressos i despeses, comandes per empleat i així un llarg etcètera. Amb aquesta informació disponible esdevindrem al coneixement, és a dir, els responsables del bar o restaurant tindran la capacitat de prendre decisions a partir d'aquesta informació. Decisions que aportaran valor a l'establiment en concret i oferir un millor servei al client, així augmentant els ingressos del bar o restaurant.

\subsection{Relació amb el client}

L'última component té com a objectiu crear un fort vincle entre l'establiment i el client que hi acudeix. Qualsevol usuari d'aquesta aplicació podrà buscar l'establiment que desitgi i veure informació sobre ell com imatges, plats més demanats, valoracions i més informació que li permeti conèixer tot respecte al bar o restaurant sense necessitat d'anar-hi presencialment. Com usuari o client d'aquest establiment, tindrà la possibilitat de demanar la comanda via plataforma amb un simple escaneig d'un codi QR que haurà col·locat a cada una de les taules. Un cop acabada la visita podrà valorar el servei acompanyat de comentaris i imatges de suport per així millorar la comunitat d'usuaris de l'aplicació.
\\\\
Aquesta component aportarà flexibilitat i comoditat per a tot tipus d'usuari de l'establiment. L'atendran més ràpidament, podrà realitzar la comanda sense pressa i rumiar ben bé que és el que al final es voldrà demanar.

%----------------------------------------------------------------------------------------
%	SECTION 4
%----------------------------------------------------------------------------------------

\section{Obstacles i riscos}

Durant tot projecte, sigui de curta o llarga durada, sempre poden sorgir un seguit d'obstacles i imprevistos que poden condicionar el resultat final d'aquest, ja que té un temps limitat per construir-lo. En cas del Treball Final de Grau, estem parlant d'uns quatre mesos aproximadament.
\\\\
El principal obstacle, que pot esdevenir i serà clau de controlar, serà la gestió del temps. El Treball Final de Grau s'ha de cursar en un període limitat i s'ha d'intentar ajustar a aquest. Caldrà fer una planificació temporal el més realista possible i anar fent punts de control periòdics perquè no es descontroli l'abast del projecte i que la planificació s'adeqüi a la realitat. En cas que en aquests controls ens adonem que s'està desviant en excés es prendrà mesures segons el cas per poder-ho remeiar.
\\\\
Un altre possible obstacle que pot aparèixer és el desconeixement d'algunes tecnologies necessàries per al desenvolupament del projecte, tecnologies que s'explicaran amb més detall en capítols posteriors. Hi ha característiques i funcionalitats que tindrà el sistema que requereixen uns coneixements tècnics per a poder-les fer realitat, les quals no disposa l'autor del projecte en nivell expert. Pot donar-se el cas també que en realitzar la planificació temporal d'una tasca es calculi un temps X que finalment no es podrà complir donat el desconeixement exacte del cost corresponent, ja que no es desconeixia la tecnologia.
\\\\
Un altre inconvenient o imprevist que pot sorgir és una gran quantitat d'errors en el codi durant la fase d'implementació del projecte. Sempre solen aparèixer nombrosos errors durant el desenvolupament d'un projecte, però solen haver-hi alguns que requereixen una dedicació extra per solucionar-los. S'ha de saber filtrar quins errors són vitals de solucionar i quins es poden evitar i esquivar-los.
\\\\
I finalment pot donar-se el cas, tot i ser molt remot, que alguns dels serveis externs que utilitza la plataforma quedin inhabilitats durant un període que dificulti el desenvolupament i testeig de l'aplicació.
\\\\
Tots aquests possibles problemes que poden sorgir durant el període d'aquest Treball Final de Grau han de ser tractats immediatament de tal manera que no es converteixin en problemes crítics difícils de solucionar. El pla d'acció per solucionar-los en cas que passin es comentarà de forma més detallada en capítols posteriors.

%----------------------------------------------------------------------------------------
%	SECTION 5
%----------------------------------------------------------------------------------------

\section{Metodologia i rigor}

Durant el transcurs de la carrera, i en especial l'especialitat d'Enginyeria del Software, he après un seguit de metodologies de treball però, sense dubte, em quedaria amb tot el grup de tècniques que engloben les metodologies àgils. En conseqüència, aquest projecte seguirà una metodologia àgil, en concret una inspirada en la metodologia \textit{Scrum}.
\\\\
Inicialment, en les primeres setmanes esdevindrà la Fase Inicial del projecte, que ve a ser l'assignatura de Gestió de Projectes (\textit{GEP}). En ella s'especificarà tot el necessari per al sistema que es vol construir. En concret, s'especificarà per un cantó la definició de l'abast i la contextualització del projecte, per altra banda la planificació temporal i finalment la gestió econòmica i de sostenibilitat. A més a més, per tal de situar-se en el punt inicial i poder fer una petita prova de com pot arribar a ser la lectura final, es realitzarà una presentació de cinc minuts explicant el realitzat en aquesta fase inicial.
\\\\
Un cop tot especificat vindrà la fase intermèdia del projecte, dit d'altra manera, la fase de desenvolupament d'aquest. En ella es crearan un seguit d'\textit{Sprints}, seguint la metodologia àgil, d'unes dues setmanes de duració que s'analitzaran en una retrospectiva posterior amb el director del projecte. Amb aquests \textit{Sprints} serà fàcil validar la feina feta i veure com va el projecte, és a dir, si la planificació inicial realitzada s'adequa al pas del projecte. Cada una d'aquests \textit{sprints} contindrà un conjunt d'\textit{Històries d'usuari} que representaran la feina a realitzar durant aquelles dues setmanes. Una \textit{història d'usuari} es pot entendre com una petita part funcional d'un projecte, és a dir, una funcionalitat d'aquest. La gràcia de dividir-ho en històries d'usuari és poder separar tota la feina que comporta un projecte en tasques molt petites i, a priori, independent entre elles. A més a més, a cada una d'aquestes històries d'usuari se li atribuirà una puntuació que representarà el cost d'implementació d'aquestes, per així poder prioritzar-les segons la disponibilitat del moment.
\\\\
A la vegada es registrarà totes les accions que es van fent per controlar quant temps es tarda a realitzar cada una de les tasques que es planteja fer. Així, en cada una de les retrospectives, es podrà analitzar si les puntuacions atorgades a cada una de les targetes o bé històries d'usuari són correctes o no.
\\\\
S'estipularà una nomenclatura i unes convencions fixes durant el desenvolupament del projecte per així tenir un millor control de quines històries d'usuari s'estan realitzant en aquell precís moment i, en general, disposar d'un historial de desenvolupament amb un format únic. Així doncs, en finalitzar la implementació es disposarà d'un codi net i que simplement llegint-ho s'entendrà el seu funcionament.

\subsection{Metodologia de desenvolupament: convencions de git}
\textbf{Commits}
\\
Missatges en anglès i en minúscules començant per un verb comú com per exemple "\textit{add}", "\textit{remove}", "\textit{update}", "\textit{refactor}", "\textit{fix}" o un qualsevol altre, que descrigui en poques paraules l'acció realitzada en el commit, així serà clar el que s'ha modificat sense necessitat d'entrar a veure els canvis.
\\\\
\textbf{Feature branch workflow}
\\
Model de desenvolupament basat en les "\textit{branches}" (branques) de git. Ajuda a realitzar un desenvolupament incremental del producte, fomenta la cronologia del codi i minimitza els possibles conflictes entre versions quan es desenvolupen. S'utilitza un repositori central com a història oficial del projecte, història que funciona sense cap tipus d'error. Internament, s'usen 2 tipologies de "\textit{branch}":
\begin{itemize}
\item \textit{Master branch}: és on està l'historial de codi acceptat i acabat del projecte. És única. Tot funciona dins d'aquesta branca.
\item \textit{Feature branches}: branches separades de la master on es realitzen els commits per desenvolupar una història d'usuari del projecte o per resoldre un problema oposat. El nom ha de ser descriptiu sobre el que s'està desenvolupant perquè al llegir-ho es sàpiga què es tracta en aquesta branch.
\end{itemize}
Una vegada acabat el desenvolupament en un feature branch s'inicia una Pull request, explicat posteriorment, per ajuntar-la amb la branca principal anomenada master. Així doncs aconsegueixes separar la part de codi funcional amb la que ho acabarà sent.
\\\\
\textbf{Pull requests}
\\
Una pull request és una conversa sobre els canvis realitzats en una branch abans d'ajuntar-ho amb la base genèrica de codi que és la branch master. Això afavoreix una revisió del codi per evitar possibles problemes amb aquell fragment modificat o afegit. I a més poder documentar d'alguna manera el transcurs o evolució del codi.

\subsection{Metodologia de desenvolupament: gestió de tiquets}
Cada vegada que s'hagi detectat algun problema en el codi de l'aplicació que està en la branca principal master s'obrirà un "\textit{issue}" en GitHub, portal que s'explicarà amb deteniment en capítols posteriors. Per tancar-ho o esmentar-ho s'aprofitarà la sintaxi pròpia de GitHub amb els commits:
\begin{itemize}
\item \textit{Closes \#12} (on 12 és el nombre del issue corresponent). Amb un commit que té aquest missatge aconseguim tancar el "\textit{issue}" determinat.
\item \textit{\#12} (on 12 és el nombre del issue corresponent). Amb un missatge d'aquest estil aconseguim referir a el "issue" indicat.
\end{itemize}

\subsection{Metodologia de desenvolupament: trello}
Per gestionar el projecte seguirem Trello a causa de la seva semblança a les metodologies àgils. El flux de les històries d'usuari passarà per dos taulers dins l'aplicació Trello:
\begin{itemize}
\item \textit{Backlog}: tauler únic durant tot el projecte. És el punt d'entrada de qualsevol requisit del projecte. Té 3 llistes, per les quals les històries han d'anar passant abans de poder ser assignades a un sprint. Aquestes llistes corresponen a les diferents fases d'especificació fins que és cada una de les històries d'usuari està totalment definida per tal de ser llançada a algun dels sprints del projecte.
\begin{itemize}
\item Història d'usuari genèrica
\item Història d'usuari amb criteris d'acceptació
\item Història d'usuari amb criteris d'acceptació i punts d'història
\end{itemize}
\item \textit{Taulers de Sprint}: taulers propis, per cada un dels cinc sprints que tindrà el projecte, on s'afegeixen les històries a realitzar que s'hagin decidit en el \textit{Sprint Planning}, considerant aquest concepte com el fet de decidir quines de les històries d'usuari del backlog van a aquesta iteració. Aquestes històries han de sortir del backlog, de la llista d'històries estimades. Per facilitar i uniformar la creació d'aquests taulers, existeix una plantilla de tauler de trello del que es realitzarà una còpia per crear cada tauler. Una vegada copiat es personalitza segons les necessitats d'aquest sprint durant el \textit{Sprint Planning}. Aquesta plantilla podrà ser modificada a final de cada sprint en la retrospectiva, per poder millorar el flux de treball intern de cada sprint.
\end{itemize}

\subsection{Metodologia de desenvolupament: codi Java}
Per desenvolupar en Java, ja que la tecnologia Android s'escriu en aquest llenguatge de programació, es seguirà una convenció de noms comú en Java segons estipula la comunitat de desenvolupadors.
\begin{itemize}
\item \textit{Funcions i variables}: Escrites en el format CamelCase\cite{camelcase} amb la primera lletra en minúscula.
\item \textit{Constant}: Escrites en majúscules i separant paraules amb “\_”.
\item \textit{Classes}: Escrites en el format CamelCase amb la primera lletra en majúscula.
\item \textit{Paquets}: Escrits en minúscula i una sola paraula.
\item \textit{Documentació}: Seguirem el format de JavaDoc\cite{javadoc}.
\end{itemize}