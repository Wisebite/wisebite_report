% Chapter Template

\chapter{Definició de l'abast} % Main chapter title

\label{Chapter3} % Change X to a consecutive number; for referencing this chapter elsewhere, use \ref{ChapterX}

Un sistema d’aquestes característiques podria tenir infinitat de requisits i funcionalitat, és per això que és important definir l’abast d’aquest projecte.

%----------------------------------------------------------------------------------------
%	SECTION 2
%----------------------------------------------------------------------------------------

\section{Objectius}

L’objectiu principal d’aquest treball final de grau és dissenyar i construir un sistema que aconsegueixi fer la vida més fàcil als empleats de qualsevol establiment de restauració i crear una millor experiència per a qualsevol usuari d’aquests locals.
\\\\
Com ja s’ha comentat anteriorment, l’elevat cost d’implantar un sistema com aquest és la dificultat més gran. És per això que un altre objectiu important és poder construir una plataforma plenament genèrica en què la puguis personalitzar al teu gust i, en definitiva, fer-la teva. Així aconseguiràs reduir una quantitat abismal en costos i obtenir majors beneficis gràcies a les funcionalitats d’aquest sistema.

%----------------------------------------------------------------------------------------
%	SECTION 3
%----------------------------------------------------------------------------------------

\section{Abast}

El sistema final estarà format per tres components molt importants que li donaran valor al producte resultant, i marcarà la diferència respecta la competència del mercat.

\subsection{Gestió de comandes}

En primer terme, el sistema serà capaç de gestionar les comandes d’un establiment de restauració. La plataforma tindrà la capacitat de crear menús i plats personalitzats, amb les preferències i opcions desitjades. Com a empleat d’aquest, podrà crear comandes facilment i sense cap problemàtica. Totes les peticions seran rebudes a cuina amb una interfície còmoda i agradable per tal d’agilitzar el procés el màxim possible.
\\\\
Implantant d’aquest component del sistema s’estarà aconseguint una millora notable en l’eficiència de les comandes. Això provocarà una satisfacció per part de la clientela, que esdevindrà a uns majors ingressos per a l’establiment.

\subsection{Anàlisi de l'establiment}

L’avantatge més important, i amb diferència, d’emmagatzemar les dades digitalment és la facilitat de realitzar un estudi detallat d’aquestes dades. El sistema tindrà la capacitat de convertir aquestes dades sense massa significat a priori a una font d’informació que serà de gran utilitat per als responsables de l’establiment.
\\\\
De forma periòdica, el sistema reportarà resums on es reflectirà informació de gran valor per a l’establiment. Informació com pot ser el tràfic setmanal, els plats més demanats, ingressos i despeses, comandes per empleat i així un llarg etcètera. Amb aquesta informació disponible esdevindrem al coneixement, és a dir, els responsables del bar o restaurant tindran la capacitat de prendre decisions a partir d’aquesta informació. Decisions que aportaran valor a l’establiment en concret i oferir un millor servei al client.

\subsection{Relació amb el client}

L’última component té com a objectiu crear un fort vincle entre l’establiment i el client. Qualsevol usuari d’aquesta aplicació podrà buscar l’establiment que desitgi i veure informació sobre ell com imatges, plats més demanats, valoracions i més. Com usuari o client d’aquest establiment, tindrà la possibilitat de demanar la comanda via plataforma amb un simple escaneig d’un codi QR. Un cop acabada la visita podrà valorar el servei acompanyat de comentaris i imatges de suport.

%----------------------------------------------------------------------------------------
%	SECTION 4
%----------------------------------------------------------------------------------------

\section{Obstacles i riscos}

El principal obstacle, que pot esdevenir i serà clau de controlar, serà la gestió del temps. El treball final de grau s’ha de cursar en un període limitat i s’ha d’intentar ajustar a aquest. Caldrà fer una planificació temporal el més realista possible i anar fent punts de control periòdics per no descontrolar l’abast del projecte i que la planificació s’adeqüi a la realitat.
\\\\
Un altre possible obstacle que pot aparèixer és el desconeixement d’algunes tecnologies necessàries per al desenvolupament del projecte. Hi ha característiques i funcionalitats que tindrà el sistema que requereixen uns coneixements tècnics per a poder-les fer realitat, les quals no disposa l’autor del projecte en nivell expert.

%----------------------------------------------------------------------------------------
%	SECTION 5
%----------------------------------------------------------------------------------------

\section{Metodologia i rigor}

Aquest projecte seguirà una metodologia àgil, en concret la metodologia Scrum.
\\\\
Inicialment, les primeres setmanes esdevindrà la Fase Inicial del projecte, que ve a ser l’assignatura de Gestió de Projectes (GEP). En ella s’especificarà tot el necessari per al sistema que es vol construir.
\\\\
Un cop tot especificat vindrà la fase intermèdia del projecte, dit d’altra manera, la fase de desenvolupament d’aquest. En ella es crearan un seguit d’Sprints, seguint la metodologia àgil, d’unes dues setmanes de duració que s’analitzaran en una retrospectiva amb el director del projecte. Amb aquests Sprints serà fàcil validar la feina feta i com va el projecte, és a dir, si la planificació inicial realitzada s’adequa al pas del projecte.
\\\\
Juntament, amb l’ús del control de versions Git, s’estipularà una nomenclatura fixa durant el desenvolupament del projecte per així tenir un millor control de quines històries d’usuari s’estan realitzant en aquell precís moment. Per cada història d’usuari a voler desenvolupar i per cada petita issue, o qüestió a canviar, que es vulgui modificar es crearà una branch per tractar en específic el tema, entenent branch com la funcionalitat que disposa un controlador de versions, és a dir, poder desenvolupar una part del projecte sense afectar el funcionament de la resta. Un cop validat que la modificació és satisfactòria, s’incorporarà a la branca principal master.
\\\\
A la vegada es registrarà totes les accions que es van fent per controlar quant temps es tarda a realitzar cada una de les tasques que es planeja fer. Així, en cada una de les retrospectives, es podrà analitzar si les puntuacions atorgades a cada una de les targetes (històries d’usuari) són correctes.